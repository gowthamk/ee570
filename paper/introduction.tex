\section{Introduction}
\label{sec:introduction}

Propositional and predicate (first-order) logics are highly expressive and
widely accepted formalisms for knowledge representation. While the former forms
the basis for various program logics (eg: Hoare logic) used in specifying
computer programs, expressivity provided by the later is sufficient to
rigorously define arithmetic (eg: Peano arithmetic). It follows that proving
theorems in formalisms like Peano arithmetic reduces to deciding the validity of
formulas in propositional and predicate logic. Does there exist an algorithm
(called a decision procedure) that can make such decisions for propositional and
predicate logics? While the answer for propositional logic is yes, it is no for
first-order logic. Nevertheless, it is possible to construct a decision
procedures for various subsets of predicate logic which are strictly weaker than
full first-order logic. Conversely, it is also possible to construct decision
procedures for propositional logic strengthened with more constraints. The rest
of this paper reviews three papers. While the first two attempt to construct
decision procedures for strengthened and weakened propositional and predicate
logic respectively, the last paper is an interesting application of first-order
logic in natural language processing and serves as a testimony to the
expressivity of the logic. 

