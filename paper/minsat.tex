\section{Minimum Satisfiability Problem}
\label{sec:minsat}

\subsection{Introduction}
Given a formula in propositional logic, satisfiability problem (SAT) asks
whether there exists an assignment to literals such that the formula evaluates
to true. SAT problem is one among the first problems to be proven NP-Complete.
Nevertheless, there exist many decision procedures, which make use of linear
time conversion of formula to conjunctive normal form (CNF), clause resolution,
and other such heuristics, to effeciently solve SAT.  Since many NP-hard
problems of practical significance can be reduced to SAT problem, an efficient
decision procedure for SAT is imperative for effeciently solving those problems.

MinSAT is an extension of SAT problem, where a formula (in CNF form) contains
two types of clauses - hard and soft. The aim is to find an assignment to
literals that satisfies all the hard clauses and minimizes the number of
satisfied soft clauses. SAT is a special case of MinSAT where there are only
hard clauses. The solution for MinSAT problem might result in a satisfying
assignment that is completely different from that of its corresponding SAT
problem. For instance, for the CNF formula:
\begin{center}
  \(
    (x1 \, \disj \, x2) \; \conj \; (x2 \, \disj \, x3)
 \)
\end{center}
Assigning \emph{true} to all literals is a solution to SAT, whereas, together
with soft clauses $x1 \conj x2 \conj x3$, solution to MinSAT assigns
\emph{false} to $x1$ and $x3$, and \emph{true} to $x2$ since this assignment
results in minimum number of satisfied soft clauses after satisfying all hard
clauses.

Weighted Partial MinSAT problem is MinSAT problem where each soft clause is
associated with certain weight. The objective is to minimize the sum of weights
of satisfied soft clauses. The paper (\cite{minsat}) proposes a branch-and-bound
algorithm, called \emph{MinSatz}, for Weighted Partial MinSAT equipped with a
novel bounding technique, and reports on an emperical investigation. In the next
sub-section, we cover some preliminaries and introduce the MinSatz algorithm as
presented in the paper. The following sub-section is an original critique of the
algorithm, and the final sub-section discusses our experience in implementing
the MinSATZ algorithm.

\subsection{MinSatz}
A literal is a propositional variable or a negated propositional variable. A
clause is a disjunction of literals. A weighted clause is a pair $(c, w)$, where
$c$ is a clause and $w$, its weight, is a natural number or infinity. A clause is
hard if its weight is infinity; otherwise it is soft. A Weighted Partial MinSAT
(MaxSAT) instance is a multiset of weighted clauses $\phi =
\{(h_1,\infty),...,(h_k,\infty),(c_1, w_1),...,(c_m, w_m)\}$, where the first k
clauses are hard and the last m clauses are soft. The Weighted Partial MinSAT
problem for instance $\phi$ consists in finding an assignment with minimum cost
satisfies all hard clauses. MinSAT problem is Weighted Partial MinSAT problem
where weights of all soft clauses is 1. Since solution for MinSAT can be
trivially extended to include weights, the paper as well as this review focuses
on the exact MinSAT problem. In the following discussion, definitions for
\emph{clique}, \emph{maximum clique} and \emph{clique partition} for an
undirected graph follow the usual definitions and reader is referred to the
paper for formal definitions.

MinSatz can be seen as an extension to the famed Davis-Putnam-Logemann-Loveland
(DPLL \cite{dpll}) algorithm for propositional CNF-SAT solving. Algorithm 1
shows the pseudo-code for DPLL. 
\begin{algorithm}
 \SetAlgoLined 
 \KwData{$\phi$ : Set of CNF clauses\\
  $\quad\quad\;\Gamma$ : Current assigment to variables}
 \KwResult{A (possibly empty) model for the formula.}
 ($\phi$,$\Gamma$) := unitPropagate ($\phi$,$\Gamma$)\;
 \If{$\phi$ contains empty clause}{return \{\}}
 \If{$\phi$ is empty}{return $\Gamma$}
  v := selectVariable ($\phi$)\;
  \Return (DPLL ($\phi \union $ \{v\},$\Gamma$) $\disj$ 
          DPLL ($\phi \union $ \{$\neg$v\},$\Gamma$))
 \caption{DPLL ($\phi$,$\Gamma$)}
\end{algorithm}
DPLL is, essentially, a backtracking search algorithm aided by a procedure
called unit propagation. A unit clause is a CNF clause that contains a single
literal ($l$). Unit propagation relies on the fact that there is only one way to
satisfy a unit clause. For each unit clause in the set of clauses, unit
propagation creates a new assignment necessary to satisfy that clause and
propagates that assignment in the following way. Clauses containing literal $l$
are removed from the set of clauses since they are already satisfied. On the
other hand, every clause containg literal $\bar{l}$ is replaced with a clause
with literal $\bar{l}$ removed. 

MinSatz extends DPLL with lower bound on the cost of satisfied soft constraints.
The algorithm, as described in the paper, is given below:
\begin{algorithm}
 \SetAlgoLined 
 \KwData{$\phi$ : Set of hard and soft CNF clauses\\
  $\quad\quad\;$ LB : Lower bound (Lowest cost incurred so far)}
 \KwResult{Minimum cost of satisfied soft constraints}
 ($\phi$) := hardUnitPropagate ($\phi$)\;
 \If{$\phi$ contains empty clause}{return -1}
 \If{$\phi$ is empty $\disj$ $\phi$ only contains empy soft clauses}
    {return n-empty($\phi$)}
 UB := n-empty($\phi$) + overestimation($\phi$)\;
 \If{UB $\le$ LB} {\Return LB}
 v := selectVariable ($\phi$)\;
 LB := MinSatz ($\phi \union $ \{v\},LB)\;
 LB := MinSatz ($\phi \union $ \{$\neg$v\},LB)\;
 \Return LB
 \caption{MinSatz ($\phi$,LB)}
\end{algorithm}
As with DPLL, MinSatz starts with unit propagation, but restricts it to hard
clauses as soft clauses need not be necessarily satisfied. MinSatz returns if
hard unit propagation resulted in contradiction or no more empty clauses left.
Otherwise, it  computes an upper bound of the maximum number of soft clauses
that are falsified (UB) if the current partial assignment is extended to a
complete one. This number UB is then compared with the number of clauses
falsified in the best assignment found so far (LB). If UB $\le$ LB, a better
solution cannot be found from the current node, and MinSatz backtracks.
Otherwise, it proceeds similar to DPLL, but while updating the lower bound with
the best solution found so far.

To compute UB, MinSatz uses a novel method to over-estimate the number of
satisfiable soft clauses. To capture the intuition, consider the two soft
clauses containg literals $l$ and $\bar{l}$. It it easy to see that both cannot
be simultaneously unsatisfiable. Now, consider the case where there are two unit
soft clauses with literals $l_1$ and $l_2$, respectively, and a hard clause
\{$l_1$,$l_2$\}. Again, both soft clauses cannot be simultaneously unsatisfiable
as that results in unsatisfiability of a hard clause. MinSatz constructs a graph
G that contains a vertex for every soft clause, and an edge between two vertices
if corresponding soft clauses cannot be simultaneously unsatisfiable (as
described above). Observe that number of cliques in the clique partition of G
gives an upper bound on number of soft clauses that can be rendered
unsatisfiable if the current assignment is completed. 

\subsection{Critique of MinSatz}
A closer look at MinSatz reveals some cases where the algorithm seems to deviate
from the expected behaviour:

\subsubsection{Case when all soft clauses are empty}
Consider the example where CNF formula consists of following hard part:
\begin{center}
  \(
    (x1 \disj x2) \conj (\neg x1 \disj x2) \conj (x1 \disj \neg x2) \conj
    (\neg x1 \disj \neg x2)
  \)
\end{center}
Assume that soft part contains a single clause $\neg x1$. Clearly, the hard part
of the formula is unsatisfiable, therefore MinSatz is expected to return -1
(signaling error) or 0 (denoting that no soft clauses can be satisfied).
However, MinSatz returns 1. This anamoly arises as MinSatz does not confirm if
the current assignment can be completed to satisfiability once it finds that all
soft clauses are empty (line number 5 of the algorithm). For the current
example, in the recursive call where $x1$ is set to \emph{true}, the sole soft
constraint is unsatisfied and MinSatz immediately returns 1. Further
recursive calls where $x1$ is set to \emph{false} have 1 as the lower bound,
which cannot be exceeded. As a result, MinSatz concludes that maximum number of
unsatisfiable soft clauses is 1. 

Notwithstanding this unexpected behaviour on CNF formulas, there is a
well-defined sub-class of CNF formulas where approach taken by MinSatz can be
proven correct, albeit in a non-trivial way. Such class is the class of CNF
formulas with Horn clauses. A Horn clause is a clause that contains atmost one
positive literal (a non-negated variable). A well-known result states that a Horn
CNF forumla without a unit clause is always satisfiable. Since hard unit
propagation results in a formula that does not contain unit clauses, if MinSatz
is invoked on Horn CNF formula, it is always the case that there exists a
satisfying assignment that completes the current assignment. It is therefore
sound to return immediately, without checking the satisfiability of hard part,
if there are no non-empty soft clauses.

\subsection {Case when first recursive call encounters unsatisfiablility}
Consider a point in the execution of MinSatz where lower bound found so far is a
positive number (k). Assume that first recursive call made (at line 13)
determines that the formula is unsatisfiable. It, therefore, returns -1, which
becomes new lower bound for second recursive call (line number 14).
Unfortunately, the information of correct lower bound is now lost. It is
therefore quite conceivable that second recursive call finds the formula
satisfiable and returns a number (k') less than k as result, which is definitely
not optimal as previous recursive calls have already found a better solution (k > k').

\subsection {Case when multiple satisfying assigments possible}
Consider the example where CNF formula consists of following hard part:
\begin{center}
  \(
    (x1 \disj x2) \conj (\neg x1 \disj x3) 
  \)
\end{center}
Let soft part of the formula contain three unit clauses - $\neg x1$, $\neg x2$,
and $\neg x3$. Consider the case when $selectVariable$ selects variable $x1$ and
it is set to true in the subsequent recursive call. As described previously,
unit propagation in the the recursive call immediately removes both hard clauses
(since both are satisfied) and returns a formula with empty hard part. Since
there are no more hard clauses, MinSatz returns 1 as only one soft constraint is
unsatisfied under current assignment of \{$x1 \mapsto true$ \}. This is clearly
sub-optimal as $x2$ and $x3$ can be interpreted in a way so as to make rest of
the soft constraints unsatisfiable as well. Therefore, correct answer in this
case should be 3.

\subsection {MinSatzEE570}
We have corrected the anamolies described in previous section and constructed
MinSatzEE570. Algorithm 3 describes MinSatzEE570.
\begin{algorithm}
 \SetAlgoLined 
 \KwData{$\phi$ : Set of hard and soft CNF clauses\\
  $\quad\quad\;$ LB : Lower bound (Lowest cost incurred so far)}
 \KwResult{Minimum cost of satisfied soft constraints}
 ($\phi$) := hardUnitPropagateEE570 ($\phi$)\;
 \If{$\phi$ contains empty clause}{return -1}
 \If{$\phi$ is empty} {return n-empty($\phi$)}
 \If{all soft clauses in $\phi$ are empty} {
   \eIf {DPLL(hard($\phi$),\{\}) returns non-empty model}
      {return n-empty($\phi$)}
      {return -1}
 }
 UB := n-empty($\phi$) + overestimation($\phi$)\;
 \If{UB $\le$ LB} {\Return LB}
 v := selectVariable ($\phi$)\;
 LB1 := MinSatz ($\phi \union $ \{v\},LB)\;
 LB  := (LB1$\le$0)?LB:LB1\;
 LB2 := MinSatz ($\phi \union $ \{$\neg$v\},LB)\;
 \eIf{LB2 = -1 $\conj$ LB1 = -1}{\Return -1}
  {\Return (LB2$\le$0)?LB:LB2}
 \caption{MinSatzEE570 ($\phi$,LB)}
\end{algorithm}
MinSatzEE570 uses a modified unit propagation procedure
(hardUnitPropagateEE570) that forces MinSatz to generate interpretation for
\emph{every} variable in hard part regardless of its relevance to satisfiability
of hard clauses. When propagating a unit clause containing literal $l$,
hardUnitPropagateEE570 behaves like hardUnitPropagate for clauses containing
$\bar{l}$. However, for a clause ($C$) containing $l$ along with $k$ other
literals ($l_1$,$l_2$,...,$l_k$), hardUnitPropagateEE570 removes $C$, but adds
k clauses of form $l_i \disj \bar{l_i}$, where $1\le i \le k$. The newly added
clauses are tautologies and do not contrain the satisfiability of formula.
However, they preserve the variables that would have been lost otherwise,
letting MinSatz to assign interpretation that leads to least cost assignment.

When all soft clauses in the formula are empty, but hard part is not yet
satisfied, it should be observed that MinSAT reduces to SAT. MinSatzEE570,
therefore calls DPLL on the remaining hard part and returns appropriate value
depending on whether DPLL was able to find a model. Lastly, the lower bound
calculation is adjusted for the fact that MinSAT returns -1

\subsection{Implementation}
We have implemented DPLL and MinSatzEE570 in racket (a dialect of Scheme). A
generic term rewrite system described in \cite{rewrite} has been used to
simplify formula during unit propagation. We evaluated MinSAT by varying the
schemes of over-estimation between 1. A naive scheme which assumes all
soft-constraints that are not yet satisfied or unsatisfied, to be satisfiable,
and 2. A more intelligent scheme based on clique construction as described in
the paper. Since DIMACS graph dataset used for evaluation in the paper requires
large effort to parse and encode graphs as propositional formulas, we used
relatively small and contrived examples for evaluation. We observed that second
scheme of over-estimation marginally outperforms first scheme for examples with
high number of mutually unsatisfiable soft constraints. Both the source code and
data are available online for reader's peruse
\footnote{https://github.com/gowthamk/ee570}.

