\section{Minimum Satisfiability Problem}
\label{sec:minsat}

\subsection{Introduction}
Given a formula in propositional logic, satisfiability problem (SAT) asks
whether there exists an assignment to literals such that the formula evaluates
to true. SAT problem is one among the first problems to be proven NP-Complete.
Nevertheless, there exist many decision procedures, which make use of linear
time conversion of formula to conjunctive normal form (CNF), clause resolution,
and other such heuristics, to effeciently solve SAT.  Since many NP-hard
problems of practical significance can be reduced to SAT problem, an efficient
decision procedure for SAT is imperative for effeciently solving those problems.

MinSAT is an extension of SAT problem, where a formula (in CNF form) contains
two types of clauses - hard and soft. The aim is to find an assignment to
literals that satisfies all the hard clauses and minimizes the number of
satisfied soft clauses. SAT is a special case of MinSAT where there are only
hard clauses. 

