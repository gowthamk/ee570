%File: main.tex
\documentclass[letterpaper]{article}
\usepackage{aaai}
\usepackage{times}
\usepackage{helvet}
\usepackage{courier}
\usepackage[linesnumbered]{algorithm2e}
\usepackage{mathpartir}
%Formatting Commands
%-------------------
\newcommand{\conj}{~\wedge~}
\newcommand{\disj}{~\vee~}
\newcommand{\union}{~\cup~}
\newcommand{\E}[1]{~\exists~{#1}}
\newcommand{\V}[1]{~\forall~{#1}}
\newcommand{\RULE}[2]{\frac{\begin{array}{c}#1\end{array}}
                           {\begin{array}{c}#2\end{array}}}
\newcommand{\entails}[2]{{#1} \models {#2}}
\newenvironment{smathpar}{
\begin{nop}\small\begin{mathpar}}{
\end{mathpar}\end{nop}\ignorespacesafterend}

\frenchspacing
\setlength{\pdfpagewidth}{8.5in}
\setlength{\pdfpageheight}{11in}
\pdfinfo{
/Title (EE570 Term Paper)
/Author (Gowtham Kaki)}
\setcounter{secnumdepth}{1}  

\begin{document}

\title{EE570 Term Paper}
\author{Gowtham Kaki\\
gkaki@cs.purdue.edu\\
}

\maketitle
\begin{abstract}
\begin{quote}
This paper is a review of three papers from recent artificial intelligence
research literature that are connected by the common theme of knowledge
representation. The first paper (Li C, Zhu Z, Manya F, Simon L, Minimum
Satisfiability and its Applications) defines the problem of minimum
satisfiability (MinSAT) for propositional logic and constructs an algorithm
based on informed bounded search of state space. Second paper (Pulina L,
Tacchella A, A Structural Approach to Reasoning with Quantified Boolean
Formulas) attempts to construct an efficient decision procedure for quantified
boolean formulas by alternating between search and clause resolution. The last
paper (Huang M, Shi X, Jin F, Zhu X, Using First-Order Logic to Compress
Sentences) is an interesting application of logics based on first-order logic to
compress English sentences. For evaluation we implemented the MinSatz algorithm
described in first paper utilizing the rewrite system discussed in the class.
While we found few inconsistencies in the algorithm as it was described in the
paper, our emperical results conclude that MinSatz is efficient when compared to
straightforward extensions of DPLL for MinSAT problem.
\end{quote}
\end{abstract}

\section{Introduction}
\label{sec:introduction}

Propositional and predicate (first-order) logics are highly expressive and
widely accepted formalisms for knowledge representation. While the former forms
the basis for various program logics (eg: Hoare logic) used in specifying
computer programs, expressivity provided by the later is sufficient to
rigorously define arithmetic (eg: Peano arithmetic). It follows that proving
theorems in formalisms like Peano arithmetic reduces to deciding the validity of
formulas in propositional and predicate logic. Does there exist an algorithm
(called a decision procedure) that can make such decisions for propositional and
predicate logics? While the answer for propositional logic is yes, it is no for
first-order logic. Nevertheless, it is possible to construct a decision
procedures for various subsets of predicate logic which are strictly weaker than
full first-order logic. Conversely, it is also possible to construct decision
procedures for propositional logic strengthened with more constraints. The rest
of this paper reviews three papers. While the first two attempt to construct
decision procedures for strengthened and weakened propositional and predicate
logic respectively, the last paper is an interesting application of first-order
logic in natural language processing and serves as a testimony to the
expressivity of the logic. 



\section{Minimum Satisfiability Problem}
\label{sec:minsat}

\subsection{Introduction}
Given a formula in propositional logic, satisfiability problem (SAT) asks
whether there exists an assignment to literals such that the formula evaluates
to true. SAT problem is one among the first problems to be proven NP-Complete.
Nevertheless, there exist many decision procedures, which make use of linear
time conversion of formula to conjunctive normal form (CNF), clause resolution,
and other such heuristics, to effeciently solve SAT.  Since many NP-hard
problems of practical significance can be reduced to SAT problem, an efficient
decision procedure for SAT is imperative for effeciently solving those problems.

MinSAT is an extension of SAT problem, where a formula (in CNF form) contains
two types of clauses - hard and soft. The aim is to find an assignment to
literals that satisfies all the hard clauses and minimizes the number of
satisfied soft clauses. SAT is a special case of MinSAT where there are only
hard clauses. The solution for MinSAT problem might result in a satisfying
assignment that is completely different from that of its corresponding SAT
problem. For instance, for the CNF formula:
\begin{center}
  \(
    (x1 \, \disj \, x2) \; \conj \; (x2 \, \disj \, x3)
 \)
\end{center}
Assigning \emph{true} to all literals is a solution to SAT, whereas, together
with soft clauses $x1 \conj x2 \conj x3$, solution to MinSAT assigns
\emph{false} to $x1$ and $x3$, and \emph{true} to $x2$ since this assignment
results in minimum number of satisfied soft clauses after satisfying all hard
clauses.

Weighted Partial MinSAT problem is MinSAT problem where each soft clause is
associated with certain weight. The objective is to minimize the sum of weights
of satisfied soft clauses. The paper (\cite{minsat}) proposes a branch-and-bound
algorithm, called \emph{MinSatz}, for Weighted Partial MinSAT equipped with a
novel bounding technique, and reports on an emperical investigation. In the next
sub-section, we cover some preliminaries and introduce the MinSatz algorithm as
presented in the paper. The following sub-section is an original critique of the
algorithm, and the final sub-section discusses our experience in implementing
the MinSATZ algorithm.

\subsection{MinSatz}
A literal is a propositional variable or a negated propositional variable. A
clause is a disjunction of literals. A weighted clause is a pair $(c, w)$, where
$c$ is a clause and $w$, its weight, is a natural number or infinity. A clause is
hard if its weight is infinity; otherwise it is soft. A Weighted Partial MinSAT
(MaxSAT) instance is a multiset of weighted clauses $\phi =
\{(h_1,\infty),...,(h_k,\infty),(c_1, w_1),...,(c_m, w_m)\}$, where the first k
clauses are hard and the last m clauses are soft. The Weighted Partial MinSAT
problem for instance $\phi$ consists in finding an assignment with minimum cost
satisfies all hard clauses. MinSAT problem is Weighted Partial MinSAT problem
where weights of all soft clauses is 1. Since solution for MinSAT can be
trivially extended to include weights, the paper as well as this review focuses
on the exact MinSAT problem. In the following discussion, definitions for
\emph{clique}, \emph{maximum clique} and \emph{clique partition} for an
undirected graph follow the usual definitions and reader is referred to the
paper for formal definitions.

MinSatz can be seen as an extension to the famed Davis-Putnam-Logemann-Loveland
(DPLL \cite{dpll}) algorithm for propositional CNF-SAT solving. Algorithm 1
shows the pseudo-code for DPLL. 
\begin{algorithm}
 \SetAlgoLined 
 \KwData{$\phi$ : Set of CNF clauses\\
  $\quad\quad\;\Gamma$ : Current assigment to variables}
 \KwResult{A (possibly empty) model for the formula.}
 ($\phi$,$\Gamma$) := unitPropagate ($\phi$,$\Gamma$)\;
 \If{$\phi$ contains empty clause}{return \{\}}
 \If{$\phi$ is empty}{return $\Gamma$}
  v := selectVariable ($\phi$)\;
  return (DPLL ($\phi \union $ \{v\},$\Gamma$) $\disj$ 
          DPLL ($\phi \union $ \{$\neg$v\},$\Gamma$))
 \caption{DPLL : An algorithm to decide propositional SAT}
\end{algorithm}


\input{qbf}

\input{nlp}

\end{document}

